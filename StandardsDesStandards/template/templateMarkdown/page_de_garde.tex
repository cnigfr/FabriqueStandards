\documentclass[a4paper,12pt]{article}

\usepackage[utf8]{inputenc}
\usepackage{graphicx}
\usepackage{geometry}
\geometry{margin=2.5cm}
\usepackage{titling}
\usepackage{fancyhdr}
\usepackage{color}
\usepackage{setspace}
\usepackage{lmodern}

\begin{document}

\begin{titlepage}
    \centering

    % Logos en haut
    \begin{minipage}{0.45\textwidth}
        \raggedright
        \includegraphics[width=4cm]{./ressources/logo_cnig.png}
    \end{minipage}
    \hfill
    \begin{minipage}{0.45\textwidth}
        \raggedleft
        \includegraphics[width=4cm]{./ressources/logo_sponsor.PNG} % <-- Remplacer par le logo du sponsor (optionnel)
    \end{minipage}

    \vspace{1cm} 
    % -> Si vous avez plusieurs sponsors, ils pourront être indiqués en partie 1.2,
    % -> Préférez un logo officiel en format PNG (souvent disponible sur le site du collaborateur via le service communication) plutôt qu'une image récupérée sur un moteur de recherche,
    % -> Le logo du sponsor doit être de la même hauteur que le logo du CNIG.

    % Titre haut centré
    {\Huge\bfseries Conseil national de l'information Géolocalisée\par}

    \vspace{1cm} 

    % Image centrale
    \includegraphics[width=6cm]{./ressources/illustration_standard.PNG} % <-- Remplacer par l'illustration du standard

    \vspace{1cm} 

    % Titre
    {\Large\bfseries `<Titre>` \par}

    \vspace{0.5cm}

    % Sous-titre
    {\Large Sous-titre \par}  % <-- Optionnel

    \vfill

    % Nom du groupe et version
    {\Large\itshape Standard du GT `<nom du GT>` \par}    % <-- Nom du groupe de travail CNIG
    \vspace{0.5cm}
    {\large Version `<x.x - jj mois aaaa>` \par}    % <-- Date de publication

\end{titlepage}

\newpage
\thispagestyle{empty}       % <-- Page blanche pour l'impression
\mbox{}
\newpage

\end{document}
